\begin{comment}
    每个模块已经预留了图片模块,如想使用,须使用如下命名方式  
    callback-[lab no]
    test-[lab no]-[pic no], 其中[pic no]用于同节多个图片
\end{comment}

\renewcommand{\labno}{3} % 实验序号,可变变量

\subsection{小实验题目}


% ***************************part 1*******************************
\subsubsection{算法流程 \labno}
% 在这里写算法流程

% 函数调用图片,密码学报告特色
\begin{comment}
\begin{figure}[H]
  \centering
  \includegraphics[width=0.4\textwidth]{pic/callback-\detokenize\expandafter{\labno}.png} 
  \caption{callback for lab \labno}
  % \label{fig:your_label} % if ref needed
\end{figure}
\end{comment}

% 伪代码, 若需要
% 伪代码写在这即可
% 有一个自动转换程序 使用vimL
\begin{algorithm}
\caption{示例伪代码}
\begin{algorithmic}[1]
\Require 输入:$x, y$
\Ensure 输出:$z$
\Procedure{SampleProcedure}{$x, y$}
    \State $z \gets x + y$
    \If{$z > 10$}
        \State $z \gets z - 10$
    \EndIf
    \While{$z < 20$}
        \State $z \gets z + 1$
    \EndWhile
    \For{$i \gets 1$ to $10$}
        \State $z \gets z - i$
    \EndFor
    \State \Return $z$
\EndProcedure
\end{algorithmic}
\end{algorithm}


% ***************************part 2*******************************
\subsubsection{测试样例}
% 在这里写测试样例

% 测试样例图片 
\begin{comment}
\begin{figure}[H]
  \centering
  \includegraphics[width=0.7\textwidth]{pic/test-\detokenize\expandafter{\labno}.png} 
  \caption{test for lab \labno}
  % \label{fig:your_label} 
\end{figure}
\end{comment}


% ***************************part 3*******************************
\subsubsection{讨论与思考}
% 在这里写讨论与思考

%多张子图:
\begin{comment}
\begin{figure}[!htbp]
    \vspace{-0.5cm}
    \centering
    \subfigure[  sub_caption1]{
        \includegraphics[width=6cm]{.jpg}
        %\caption{fig1}
    }

    \quad %注意空行会强制子图换行显示
    \subfigure[sub_cap2]{
        \includegraphics[width=6cm]{2.jpg}
    }
    \quad
    \subfigure[sub_cap3]{
        \includegraphics[width=6cm]{3.jpg}
    }

    \quad
    \subfigure[sub_cap4]{
        \includegraphics[width=6cm]{4.jpg}
    }
    \caption{ caption}
\end{figure}
\end{comment}