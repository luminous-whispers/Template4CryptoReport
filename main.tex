\documentclass[UTF8]{ctexart}
\usepackage{fancyhdr}
\usepackage{titlesec}
\usepackage{lipsum}
\usepackage{enumitem}
\usepackage{amsmath}
\usepackage{comment} % 注释

% 设置页边距
\setlength{\headheight}{13.06885pt}
\addtolength{\topmargin}{-0.42162pt}
\usepackage{geometry}
\geometry{
    a4paper,
    left=20mm,
    right=20mm,
    top=20mm,
    bottom=20mm,
}

% 页眉页脚
\pagestyle{fancy}
\fancyhf{}
\rhead{姓名-学号}
\lhead{第一次密码学实验报告}
\cfoot{\thepage}

% 设置段落
\usepackage{setspace} % 导入 setspace 宏包
% \onehalfspacing % 将行距设置为 1.5 倍
\doublespacing % 将行距设置为 2 倍
\setlength{\parindent}{2em} %设置首行缩进为2个字符
\setlength{\parskip}{0.5em} %设置段落间距为0.5个字符

% 添加一些好看的开源字体
\usepackage{fontspec} 
\setmainfont[Path=font/,
BoldFont = JetBrainsMono-Bold.ttf,
ItalicFont = JetBrainsMono-Italic.ttf,
BoldItalicFont = JetBrainsMono-BoldItalic.ttf]{JetBrainsMono-Regular.ttf} 
% 中文字体,不想用注释掉即可
\setCJKmainfont[Path=font/,
BoldFont = LXGWWenKai-Bold.ttf,
ItalicFont = LXGWWenKai-Light.ttf]{LXGWWenKai-Regular.ttf}
\newCJKfontfamily\yaheimono[Path=font/]{YaHeiMono.ttf}
\newCJKfontfamily\sourcehan[Path=font/, BoldFont=SourceHanSansCN-Bold.otf]{SourceHanSansCN-Regular.otf}

% 分级标题格式
\ctexset{
  section = {
    format = \Large\bfseries\sourcehan %\centering
  },
  subsection = {
    format = \large\bfseries\sourcehan
  },
  subsubsection = {
    format = \bfseries\yaheimono
  }
}

% 伪代码
\usepackage{algorithm}
\usepackage{algpseudocode}
% 以下将伪代码行号后标点改为.
\usepackage{etoolbox}
\makeatletter
\def\alglinenumber#1{\footnotesize#1.}
\patchcmd{\algocf@linesnumbered}{\footnotesize#1:}{\alglinenumber{#1}}{}{}
\makeatother

% 图片
\usepackage{graphicx}
\usepackage{caption}
\usepackage{float}


\begin{document}
% *********separate**************


\begin{center}
    {\sourcehan\Huge\textbf{实验报告}}
\end{center}


% *********separate**************
\section{实验目的}
密码学形式主义调研报告


\section{实验环境}

Windows 10
\par
Python 3.10


% *********separate**************
\section{实验内容}
% 这是字体测试
你好,这里是密码学实验报告模板

hello, this is a report template for cryptography lab

math test $x^2$


% 导入各个实验tex文件
\begin{comment}
    每个模块已经预留了图片模块,如想使用,须使用如下命名方式  
    callback-[lab no]
    test-[lab no]-[pic no], 其中[pic no]用于同节多个图片
\end{comment}

\newcommand{\labno}{1} % 实验序号,可变变量. 第一次使用new,第二次用renew

\subsection{小实验题目}


% ***************************part 1*******************************
\subsubsection{算法流程 \labno}
% 在这里写算法流程

% 函数调用图片,密码学报告特色
% \begin{comment}
\begin{figure}[H]
  \centering
  \includegraphics[width=0.8\textwidth]{pic/callback-\detokenize\expandafter{\labno}.png} 
  \caption{callback for lab \labno}
  % \label{fig:your_label} % if ref needed
\end{figure}
% \end{comment}

% 伪代码, 若需要
% 伪代码写在这即可
% 有一个自动转换程序 使用vimL
\begin{algorithm}
\caption{示例伪代码}
\begin{algorithmic}[1]
\Require 输入:$x, y$
\Ensure 输出:$z$
\Procedure{SampleProcedure}{$x, y$}
    \State $z \gets x + y$
    \If{$z > 10$}
        \State $z \gets z - 10$
    \EndIf
    \While{$z < 20$}
        \State $z \gets z + 1$
    \EndWhile
    \For{$i \gets 1$ to $10$}
        \State $z \gets z - i$
    \EndFor
    \State \Return $z$
\EndProcedure
\end{algorithmic}
\end{algorithm}


% ***************************part 2*******************************
\subsubsection{测试样例}
% 在这里写测试样例

% 测试样例图片 
% \begin{comment}
\begin{figure}[H]
  \centering
  \includegraphics[width=0.4\textwidth]{pic/test-\detokenize\expandafter{\labno}.png} 
  \caption{test for lab \labno}
  % \label{fig:your_label} 
\end{figure}
% \end{comment}


% ***************************part 3*******************************
\subsubsection{讨论与思考}
% 在这里写讨论与思考

%多张子图:
\begin{comment}
\begin{figure}[!htbp]
    \vspace{-0.5cm}
    \centering
    \begin{subfigure}[b]{0.9\textwidth}
        \includegraphics[width=\linewidth]{pic/fermat_power.png}
        \caption{费马小定理求逆+快速幂计算点数乘}
    \end{subfigure}
    \quad
    \begin{subfigure}[b]{0.9\textwidth}
        \includegraphics[width=\linewidth]{pic/egcd_power.png}
        \caption{扩展欧几里得算法求逆+快速幂计算点数乘}
    \end{subfigure}
    \quad
    \begin{subfigure}[b]{0.9\textwidth}
        \includegraphics[width=\linewidth]{pic/egcd_window.png}
        \caption{扩展欧几里得算法求逆+窗口法计算点数乘}
    \end{subfigure}

    \vspace{0.5cm}

    \begin{subfigure}[b]{0.9\textwidth}
        \includegraphics[width=\linewidth]{pic/egcd_naf.png}
        \caption{扩展欧几里得算法求逆+NAF快速幂计算点数乘}
    \end{subfigure}
    \caption{性能分析图}
\end{figure}
\end{comment} 

% *********separate**************
\clearpage
\section{实验收获与思考}
\subsection{收获}
写报告真的让我学会蛮多!!

% 这里放报告特色思考题

\subsection{思考题}
% ***************************part 1*******************************
\subsubsection{题目一:}
\textbf{题干}
\par
这里是答案

% ***************************part 2*******************************
\subsubsection{题目一:}



\end{document}

